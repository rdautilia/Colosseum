%
%  untitled
%
%  Created by Roberto D'Autilia on 2017-03-21.
%  Copyright (c) 2017 __MyCompanyName__. All rights reserved.
%
\documentclass[]{article}

% Use utf-8 encoding for foreign characters
\usepackage[utf8]{inputenc}

% Setup for fullpage use
\usepackage{fullpage}

% Uncomment some of the following if you use the features
%
% Running Headers and footers
%\usepackage{fancyhdr}

% Multipart figures
%\usepackage{subfigure}

% More symbols
%\usepackage{amsmath}
%\usepackage{amssymb}
%\usepackage{latexsym}
% ROB>>More symbols
\usepackage{amsmath}
\usepackage{amssymb}
\usepackage{amsthm}
\usepackage{latexsym}
\usepackage[colorlinks]{hyperref}
%\usepackage{tikz}
\newtheorem{definition}{Definizione}
\newtheorem{theorem}{Teorema}
\newtheorem{corollario}{Corollario}
\newtheorem{proposizione}{Proposizione}
\newtheorem{exercise}{Esercizio}
\newtheorem{domanda}{Domanda}
\newtheorem{lemma}{Lemma}
\newtheorem{esempio}{Esempio}
\renewcommand{\Re}{\mathop{\rm Re}}
\renewcommand{\Im}{\mathop{\rm Im}}
\renewcommand{\proofname}{Dimostrazione}
% ROB
\let\oldemptyset\emptyset
\let\emptyset\varnothing
% Surround parts of graphics with box
\usepackage{boxedminipage}

% Package for including code in the document
\usepackage{listings}

% If you want to generate a toc for each chapter (use with book)
\usepackage{minitoc}

% This is now the recommended way for checking for PDFLaTeX:
\usepackage{ifpdf}

%\newif\ifpdf
%\ifx\pdfoutput\undefined
%\pdffalse % we are not running PDFLaTeX
%\else
%\pdfoutput=1 % we are running PDFLaTeX
%\pdftrue
%\fi

\ifpdf
\usepackage[pdftex]{graphicx}
\else
\usepackage{graphicx}
\fi
\title{Flussi Pedonali nell'Area del Colosseo}
\author{Roberto D'Autilia}

\date{2017-03-21}

\begin{document}
\lstset{
	language=C,                             % Code langugage
	basicstyle=\ttfamily,                   % Code font, Examples: \footnotesize, \ttfamily
	breaklines=true,
	%	keywordstyle=\color{OliveGreen},        % Keywords font ('*' = uppercase)
	%	commentstyle=\color{gray},              % Comments font
	%	numbers=left,                           % Line nums position
	%	numberstyle=\tiny,                      % Line-numbers fonts
	%	stepnumber=1,                           % Step between two line-numbers
	%	numbersep=5pt,                          % How far are line-numbers from code
	%	backgroundcolor=\color{lightlightgray}, % Choose background color
	%	frame=none,                             % A frame around the code
	%	tabsize=2,                              % Default tab size
	%	captionpos=b,                           % Caption-position = bottom
	%	breaklines=true,                        % Automatic line breaking?
	%	breakatwhitespace=false,                % Automatic breaks only at whitespace?
	%	showspaces=false,                       % Dont make spaces visible
	%	showtabs=false,                         % Dont make tabls visible
	columns=flexible,                       % Column format
	%	morekeywords={__global__, __device__},  % CUDA specific keywords
}

\ifpdf
\DeclareGraphicsExtensions{.pdf, .jpg, .tif}
\else
\DeclareGraphicsExtensions{.eps, .jpg}
\fi

\maketitle


\begin{abstract}
{\tt File:\jobname.tex}\\
\end{abstract}

\section{Introduzione}
In questo lavoro vogliamo studiare i flussi pedonali nell'area del Colosseo nell'assetto urbanistico attuale e in quello di progetto.
I luoghi dai quali si muove la popolazione sono luscita della metro $M_1$, la Via Sacra $M_2$ e Via di San Gregorio $M_3$.
Abbiamo due carte, la sitazione attuale ({\tt 00-FASE0.jpg}) e il progetto ({\tt 00-FASE1.jpg}).

La mappa {\tt 00-FASE0.jpg} deve essere scalata per ssere adattata alla simulazione.
La scala ottimale è $(0.27,0.27)$.

Ora dobbiamo visualizzare le coordinate del mouse.
Poi ricostruire tutte le aree chiuse.
Le disegno in verde per non confonderle con quelle ancora non corrette.

{\bf Colosseo} lo modellizziamo come due emicicli (poligoni chiusi in modo di avere un'entrata e un'uscita).
Fatto questo cominciamo a simulare una popolazione di 1000 persone che entra nel Colosseo sulla prima entrata.
Poiché il {\sl drift} all'interno del Colosseo ci occorre anche un poligono area interna del Colosseo.
\section{Codice}
Vediamo come è fatto il codice.
Le costanti sono

@o ../../test/jl/costanti.jl @{
# INIZIALIZZO LE COSTANTI
const N=1000 ::Int64				# Il numero di pedoni
const dt = 1.0 ::Float64			# Il passo di integrazione
const numero_iterazioni = 500		# Il numero di iterazioni della simulazione
const diag = sqrt(2) ::Float64		# diagonale
const passo = 1.0 ::Float64			# La dimensione di un passo
const raggio = 0.5 ::Float64		# Il raggio di non sovrapposizione dei pedoni
const dimenpedone = 2.0 ::Float64	# La dimensione del pedone	
const scalax = 1.5 ::Float64		# lunghezza del passo di un pedone nella direzione x
const scalay = 1.5 ::Float64		# lunghezza del passo di un pedone nella direzione y
@}


@o ../../test/jl/variabili_globali.jl @{
# Variabili globali che sarebbe meglio eliminare
posizioni_prima = zeros(2,N)				# le posizioni dei pedoni al tempo t
#posizioni_dopo = rand(30.0:821.0,2,N)		# le posizioni dei pedoni al tempo t+dt
altre = [rand(0.0:0.0,1,N);rand(100.0:100.0,1,N)] # altre posizioni
posizioni_dopo = hcat([rand(100.0:100.0,1,N);rand(200.0:200.0,1,N)],altre) # le posizioni dei pedoni al tempo t+dt
velocita = zeros(2,N)						# le posizioni dei pedoni al tempo t+dt
k = 0 ::Int64								# Un iteratore
passo = 1.0 ::Float64						# La dimensione di un passo
raggio = 0.5 ::Float64						# Il raggio di non sovrapposizione dei pedoni
dimenpedone = 2.0 ::Float64						# La dimensione del pedone	
scalax = 1.5 ::Float64						# lunghezza del passo di un pedone nella direzione x
scalay = 1.5 ::Float64						# lunghezza del passo di un pedone nella direzione y

mmm = 0.0 ::Float64							# variabile di appoggio, ricontrollare se ci serve
@}

Per costruire i poligoni degli ostacoli dei pedoni costruiamo una costante di tipo  {\tt dictionary} 
@o ../../test/jl/edifici_coord.jl @{
###### DICTIONARY POLIGONI DEGLI EDIFICI ##################
#colosseo=[0 0 100 0; 100 0 100 200; 100 200 50 50; 50 50 0 100; 0 100 0 0]
###### DICTIONARY POLIGONI DEGLI EDIFICI ##################
#colosseo=[0 0 100 0; 100 0 100 200; 100 200 50 50; 50 50 0 100; 0 100 0 0]
const edifici_coord = Dict{String, Array}(
"colosseo_coord" => [314 256; 329 242; 359 223; 382 212; 419 204; 454 200; 489 204; 523 211; 558 223; 597 245; 624 266; 659 306; 682 352; 688 403; 678 452;
651 444; 631 482; 600 507; 562 525; 520 533; 485 531; 446 524; 412 510; 374 490; 344 462; 314 420; 300 384; 300 351; 304 323; 312 299; 322 285], # L'earea interna
"colosseoa_coord" => [373 203;390 198; 410 192; 440 190; 464 190; 487 191; 507 194; 537 202; 567 213; 594 229; 615 244; 642 272; 662 298; 678 325; 693 357; 698 356;
698 411; 699 444; 691 459; 679 455; 687 422; 689 402; 688 387; 687 374; 684 358; 679 344; 673 330; 664 317; 657 305; 648 292; 638 280; 627 270; 615 259; 603 250;
590 242; 577 234; 561 224; 548 220; 532 212; 518 210; 502 206; 485 204; 471 202; 455 201; 441 201; 425 202; 409 206; 397 209; 380 213], # L'emiciclo a
"colosseob_coord" => [659 449; 652 464; 643 478; 636 488; 625 498; 616 506; 605 514; 595 519; 583 524; 559 534; 544 537; 530 539; 518 541; 502 541; 489 540; 473 538; 
461 536; 444 531; 447 525; 465 528; 477 531; 491 533; 503 533; 518 533; 532 533; 545 531; 559 528; 573 523; 583 518; 593 514; 604 507; 615 502; 624 491; 631 483;
638 475; 644 464; 649 451; 650 444], # L'emiciclo b
"colosseoc_coord" => [409 519; 395 511; 381 503; 371 496; 361 488; 349 478; 340 468; 330 459; 322 448; 313 432; 309 423; 303 411; 298 400; 295 386; 294 374;
293 360; 293 344; 261 342; 264 323; 269 305; 276 289; 281 277; 288 266; 323 288; 315 303; 311 312; 307 325; 303 338; 301 350; 301 362; 300 373; 302 387;
307 400; 315 421; 329 443; 336 454; 347 466; 367 485; 388 498; 399 505; 412 510], # L'emiciclo c
"colosseod_coord" => [305 245; 321 230; 335 221; 355 223; 340 233; 326 242; 314 254], # L'emiciclo d
"g1_coord" => [139 36; 201 66; 226 71; 241 70; 270 81; 273 71; 201 57; 159 41],
"g2_coord" => [287 74; 379 93; 417 91; 431 96; 456 90; 465 98; 436 109; 432 125; 348 110; 348 102; 342 95; 284 81],
"g3_coord" => [424 83; 429 89; 459 83; 476 100; 441 116; 440 126; 518 141; 529 147; 560 159; 597 172; 632 193; 682 218; 737 246; 734 238; 703 200; 553 105; 448 76],
"arcoCostantino_coord" => [121 482; 131 439; 201 454; 191 497],
"prato1_coord" => [222 147; 230 127; 333 144; 315 182],
"prato2_coord" => [179 256; 213 171; 287 199; 248 238; 229 275],
"venereroma_coord" => [25 100; 193 165; 121 358; 25 329],
"metasudans_coord" => [120 424; 175 271; 224 291; 212 339; 222 400; 212 446],
"palatino_coord" => [23 783; 23 350; 100 377; 78 472; 80 488; 90 565; 83 652; 52 783],
"sangregorio_coord" => [120 785; 130 760; 147 740; 177 709; 203 683; 235 659; 269 639; 307 625; 353 617; 413 619; 470 620; 544 624; 649 625; 684 645; 738 785],
"bordoest_coord" => [608 230; 670 260; 700 280; 724 306; 734 330; 738 352; 737 376; 724 429; 693 504; 670 535; 622 565; 
570 573; 513 579; 451 578; 391 575; 343 576; 303 583; 262 594; 273 590; 221 614; 193 634; 157 660; 171 620; 185 580; 202 540; 206 524; 230 507; 420 561; 425 537;
503 543; 583 549; 638 536; 679 509; 702 465; 684 449; 693 402; 691 351; 676 307; 663 282; 643 254; 612 228])
@}

Abbiamo poi una funzione che crea i poligoni degli edifici
@o ../../test/jl/disegna_poligono.jl @{
######################## UNA FUNZIONE CHE CREA I POLIGONI DEGLI EDIFICI ###############
function disegna_poligono(polig_coord)
	the_shape = ConvexShape()
	set_pointcount(the_shape, size(polig_coord)[1])
	for i = 1:size(polig_coord)[1]
		set_point(the_shape, i-1, Vector2f(polig_coord[i,1], polig_coord[i,2]))
	end
	set_position(the_shape, Vector2f(0.0, 0.0))
	set_fillcolor(the_shape, SFML.transparent)
	set_outlinecolor(the_shape, SFML.green)
	set_outline_thickness(the_shape, 2)
	return the_shape
end
##########################################################################################

@}

Abbiamo poi una funzione che verifica che le coordinate del pedone non siano all'interno di alcun poligono chiuso
@o ../../test/jl/esterno.jl @{
############ UNA FUNZIONE CHE VERIFICA CHE UN PUNTO NON SIA INTERNO AD ALCUN POLIGONO ###########
function esterno(lax, lay, polig_coord)
w = map(ciclo_poligono, values(polig_coord))
a = sum(map(x->inpoly(lax,lay,x),w))
	if a>0
		return 1
	else
		return 0
	end
end
@}
#################################################################################################



 
\bibliographystyle{plain}
%\bibliography{}
%ROB
\bibliography{/Users/dautilia-n/Desktop/DTC/WorkInProgress/References/bibliografia_rob.bib} % name your BibTeX data base
%ROB
\end{document}

%%%%%%% ESEMPIO FIGURA %%%%%%%%%%%%%%%
%\begin{figure}[ht]
%\centering
%\includegraphics[width=0.8\textwidth]{CCBeta0307409_bw.pdf}
%\caption{The number of people that can be fed in an area $C$ as function of time for $\beta=0.3$ (dashed line) $\beta=0.74$ (continuous line)) and $\beta=0.9$ (dotted line)}
%\label{fig:chini}
%\end{figure}
















































