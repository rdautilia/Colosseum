% -------------------------------------------------------------------------
% ------ nuweb macros (redefine as desired, or omit with "nuweb -p") ------
% -------------------------------------------------------------------------
\providecommand{\NWtxtMacroDefBy}{Macro defined by}
\providecommand{\NWtxtMacroRefIn}{Macro referenced in}
\providecommand{\NWtxtMacroNoRef}{Macro never referenced}
\providecommand{\NWtxtDefBy}{Defined by}
\providecommand{\NWtxtRefIn}{Referenced in}
\providecommand{\NWtxtNoRef}{Not referenced}
\providecommand{\NWtxtFileDefBy}{File defined by}
\providecommand{\NWsep}{${\diamond}$}
\providecommand{\NWlink}[2]{\hyperlink{#1}{#2}}
\providecommand{\NWtarget}[2]{% move baseline up by \baselineskip 
  \raisebox{\baselineskip}[1.5ex][0ex]{%
    \mbox{%
      \hypertarget{#1}{%
        \raisebox{-1\baselineskip}[0ex][0ex]{%
          \mbox{#2}%
}}}}}
% -------------------------------------------------------------------------

%
%  untitled
%
%  Created by Roberto D'Autilia on 2017-03-21.
%  Copyright (c) 2017 __MyCompanyName__. All rights reserved.
%
\documentclass[]{article}

% Use utf-8 encoding for foreign characters
\usepackage[utf8]{inputenc}

% Setup for fullpage use
\usepackage{fullpage}

% Uncomment some of the following if you use the features
%
% Running Headers and footers
%\usepackage{fancyhdr}

% Multipart figures
%\usepackage{subfigure}

% More symbols
%\usepackage{amsmath}
%\usepackage{amssymb}
%\usepackage{latexsym}
% ROB>>More symbols
\usepackage{amsmath}
\usepackage{amssymb}
\usepackage{amsthm}
\usepackage{latexsym}
\usepackage[colorlinks]{hyperref}
%\usepackage{tikz}
\newtheorem{definition}{Definizione}
\newtheorem{theorem}{Teorema}
\newtheorem{corollario}{Corollario}
\newtheorem{proposizione}{Proposizione}
\newtheorem{exercise}{Esercizio}
\newtheorem{domanda}{Domanda}
\newtheorem{lemma}{Lemma}
\newtheorem{esempio}{Esempio}
\renewcommand{\Re}{\mathop{\rm Re}}
\renewcommand{\Im}{\mathop{\rm Im}}
\renewcommand{\proofname}{Dimostrazione}
% ROB
\let\oldemptyset\emptyset
\let\emptyset\varnothing
% Surround parts of graphics with box
\usepackage{boxedminipage}

% Package for including code in the document
\usepackage{listings}

% If you want to generate a toc for each chapter (use with book)
\usepackage{minitoc}

% This is now the recommended way for checking for PDFLaTeX:
\usepackage{ifpdf}

%\newif\ifpdf
%\ifx\pdfoutput\undefined
%\pdffalse % we are not running PDFLaTeX
%\else
%\pdfoutput=1 % we are running PDFLaTeX
%\pdftrue
%\fi

\ifpdf
\usepackage[pdftex]{graphicx}
\else
\usepackage{graphicx}
\fi
\title{Flussi Pedonali nell'Area del Colosseo}
\author{Roberto D'Autilia}

\date{2017-03-21}

\begin{document}
\lstset{
	language=C,                             % Code langugage
	basicstyle=\ttfamily,                   % Code font, Examples: \footnotesize, \ttfamily
	breaklines=true,
	%	keywordstyle=\color{OliveGreen},        % Keywords font ('*' = uppercase)
	%	commentstyle=\color{gray},              % Comments font
	%	numbers=left,                           % Line nums position
	%	numberstyle=\tiny,                      % Line-numbers fonts
	%	stepnumber=1,                           % Step between two line-numbers
	%	numbersep=5pt,                          % How far are line-numbers from code
	%	backgroundcolor=\color{lightlightgray}, % Choose background color
	%	frame=none,                             % A frame around the code
	%	tabsize=2,                              % Default tab size
	%	captionpos=b,                           % Caption-position = bottom
	%	breaklines=true,                        % Automatic line breaking?
	%	breakatwhitespace=false,                % Automatic breaks only at whitespace?
	%	showspaces=false,                       % Dont make spaces visible
	%	showtabs=false,                         % Dont make tabls visible
	columns=flexible,                       % Column format
	%	morekeywords={__global__, __device__},  % CUDA specific keywords
}

\ifpdf
\DeclareGraphicsExtensions{.pdf, .jpg, .tif}
\else
\DeclareGraphicsExtensions{.eps, .jpg}
\fi

\maketitle


\begin{abstract}
{\tt File:\jobname.tex}\\
\end{abstract}

\section{Introduzione}
In questo lavoro vogliamo studiare i flussi pedonali nell'area del Colosseo nell'assetto urbanistico attuale e in quello di progetto.
I luoghi dai quali si muove la popolazione sono luscita della metro $M_1$, la Via Sacra $M_2$ e Via di San Gregorio $M_3$.
Abbiamo due carte, la sitazione attuale ({\tt 00-FASE0.jpg}) e il progetto ({\tt 00-FASE1.jpg}).

La mappa {\tt 00-FASE0.jpg} deve essere scalata per ssere adattata alla simulazione.
La scala ottimale è $(0.27,0.27)$.

Ora dobbiamo visualizzare le coordinate del mouse.
Poi ricostruire tutte le aree chiuse.
Le disegno in verde per non confonderle con quelle ancora non corrette.

{\bf Colosseo} lo modellizziamo come due emicicli (poligoni chiusi in modo di avere un'entrata e un'uscita).
Fatto questo cominciamo a simulare una popolazione di 1000 persone che entra nel Colosseo sulla prima entrata.
Poiché il {\sl drift} all'interno del Colosseo ci occorre anche un poligono area interna del Colosseo.

Vediamo come è fatto il codice.
Le costanti sono

\begin{flushleft} \small
\begin{minipage}{\linewidth} \label{scrap1}
\protect\makebox[0ex][r]{\NWtarget{nuweb1}{\rule{0ex}{0ex}}\hspace{1em}}\verb@"../../test/jl/costanti.jl"@\nobreak\ {\footnotesize 1 }$\equiv$
\vspace{-1ex}
\begin{list}{}{} \item
\mbox{}\verb@@\\
\mbox{}\verb@\#MEMO: MODIFICARE IL 2N NEI FOR, CHE È INGUARDABILE@\\
\mbox{}\verb@\# INIZIALIZZO@\\
\mbox{}\verb@const N=1000 ::Int64                \# Il numero di pedoni@\\
\mbox{}\verb@const dt = 1.0 ::Float64               \# Il passo di integrazione@\\
\mbox{}\verb@const numero\_iterazioni = 500            \# Il numero di iterazioni della simulazione@\\
\mbox{}\verb@const diag = sqrt(2) ::Float64            \# diagonale@\\
\mbox{}\verb@@{\NWsep}
\end{list}
\vspace{-2ex}
\end{minipage}\\[4ex]
\end{flushleft}
\bibliographystyle{plain}
%\bibliography{}
%ROB
\bibliography{/Users/dautilia-n/Desktop/DTC/WorkInProgress/References/bibliografia_rob.bib} % name your BibTeX data base
%ROB
\end{document}

%%%%%%% ESEMPIO FIGURA %%%%%%%%%%%%%%%
%\begin{figure}[ht]
%\centering
%\includegraphics[width=0.8\textwidth]{CCBeta0307409_bw.pdf}
%\caption{The number of people that can be fed in an area $C$ as function of time for $\beta=0.3$ (dashed line) $\beta=0.74$ (continuous line)) and $\beta=0.9$ (dotted line)}
%\label{fig:chini}
%\end{figure}
















































